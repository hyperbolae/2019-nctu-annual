$s_0,\dots,s_{n-1}$ is a sequence of integers. 
In this problem, a subarray of a sequence is an array containing some consecutive elements of the sequence. 
\texttt{harryoooooooooo} defines the greatness of a continuous segment $s_\ell,s_{\ell+1},\dots,s_{r-1}$
as its longest subarray which forms an arithmetic sequence.
In other words, the greateness of $s_\ell,s_{\ell+1},\dots,s_{r-1}$ is the length of its longest subarray
which owns the following feature: there exist two integers $a$ and $d$ such that 
the subarray can be described as $a, a+d, a+2d, \dots, a+(m-1)d$, where $m$ denotes the length of subarray. 
To make this problem harder, \texttt{harryoooooooooo} sometimes modifies the sequence. 
He chooses a subarray and increases or decreases all elements with an integer $v$. Please find the answers for him.
