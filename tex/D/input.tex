In the first line, three integers $n, m, k$ are given. $n$ is an integer in $[1, 3000]$ that denotes the number of planets, and $m$ is an integer in $[1, n(n-1)/2]$ that denotes the number of wormholes. $k$ is an integer in $[1, 20]$ that specifies the number of queries. The $n$ planets are numbered from $1$ to $n$. Then the description of the $m$ wormholes follows. Each wormhole is specified by the identifier of the end-planets, $u$ and $v$ for some $u \ne v \in \{1, 2, \ldots, n\}$. Then the description of the $k$ queries follows. Each query gives an $(X, Y)$ pair. You may assume that the planets are connected.
